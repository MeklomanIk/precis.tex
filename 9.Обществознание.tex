\documentclass[a4paper, 12pt]{article}
\usepackage[T2A]{fontenc}
\usepackage[utf8]{inputenc}
\usepackage[english, russian]{babel}
\usepackage{indentfirst}
\usepackage{cmap}
\usepackage[lmargin={1cm}, rmargin={1cm}]{geometry}
\usepackage{multirow}
\title{Обществознание.9 класс}
\author{meklomanik(Михаил Колесников)}
\date{Лето 2021}
\begin{document} 
    \maketitle{} \clearpage
    \tableofcontents \clearpage
    \section{Политика}
        \subsection{Политика и власть}
        \subsection{Политические режимы}
        \subsection{Правовое государство}
        \subsection{Гражданское общество и государство}
        \subsection{Участие граждан в политической жизни}
        \subsection{Политические партии и движения}
        \subsection{Межгосударственные отношения}
    \section{Гражданин и государство}
        \subsection{Основы конституционного строя РФ}
        \subsection{Права и свободы человека и гражданина}
        \subsection{Высшие органы государственной власти в РФ}
        \subsection{Россия - федеративное государство}
        \subsection{Судебная система РФ}
        \subsection{Правоохранительные органы РФ}
    \section{Основы российского законодательства}
        \subsection{Роль права в жизни человека, общества и государства}
        \subsection{Правоотношения и субъекты права}
        \subsection{Правонарушения и юридическая ответственность}
        \subsection{Гражданские правоотношения}
        \subsection{Право на труд. Трудовые отношения}
        \subsection{Семья под защитой закона}
        \subsection{Административные правоотношения}
        \subsection{Уголовно-правовые правоотношения}
        \subsection{Праврвре регулирование отношений в сфере образования}
        \subsection{Международно-правовая защита жертв вооружённых конфликтов}
    \section{Словарь}
        \textbf{Адвокатура} --- объединение юристов-профессионалов(адвокатов), главной функцией которого является оказание помощи всем, кто в ней нуждается. \par
        \textbf{Административная ответственность} --- один из видов юридической ответственности, выражающийся в административном наказании за совершённое административное правонарушение(проступок). \par
        \textbf{Админстративные правонарушение(проступок)} ---противоправное, виновное действие иои бездействие, посягающее на установленный государственный или общественный порядок, на порядок управления, собственность, права и свободы граждан, за которое предусмотрена административная ответственность. \par
        \textbf{Административные правоотношения} --- один из видов правоотношений, особенность которых состоит в том, что они складываются в сфере деятельности исполнительной власти и регулируются нормами административного права. \par
        \textbf{Бандитизм} --- создание устойчивой вооружённой группы(банды) в целях нападения на граждан или организации, а равно руководство такой группой. \par
        \textbf{Брак} --- союз мужчины и женщины, имеющий целью создание семьи, заключённый в установленном законом порядке и порождающий взаимные права и обязанности супругов. \par
        \textbf{Вандализм} --- осквернение зданий или ины хсооружений, порча имущества на общественном транспорте или в иных общественных местах. \par
        \textbf{Власть} --- право и возможность распоряжаться кем-нибудь, чем-нибудь, подчинять своей воле. \par
        \textbf{Всеобщая декларация прав человека} --- выдающийся правовой документ современности, в котором впервые в истории провозглашены права человека, подлежащие всеобщему соблюдению. \par
        \textbf{Выбора} --- процедура избрания кого-либо путём голосования. \par
        \textbf{Вымогательство} --- требование передаачи чужого имущества (или права на имущество) под угрозой применения насилия над потерпевшим или над его близкими, а также под другими угрозами(например, уничтожить имущество, распространить сведения.)
        \textbf{Государство} --- форма организации политической власти, осуществляющей управление обществом и обладающей суверенитетом. \par
        \textbf{Грабёж} --- открытое(в присутствии потерпевшего или других людей) хищение чужого имущества \par
        \textbf{Гражданские правоотношения} --- общественные отношения мужду субъектами, урегулированные нормами гражданского права. \par
        \textbf{Гражданское общество} --- совокупность внегосударственных общественных отношений и ассоциаций(объединений), выражающих разнообразные интересы и потребности членов общества, при этом личность и организации граждан ограждены законами от прямого вмешательства государственной власти. \par
        \textbf{Гражданское право} --- отрасль права, представляющая собой совокупонсоть норм, которые регулируют имущественные и связанные с ними личные неимущественные отношения, основанные на автономии и имущественной самостоятельности их учатсников. \par
        \textbf{Гражданство} --- устойчивая политико-правовая связь человека с государством, предполагающая определённые праваЮ обязанности и ответственность. \par
        \textbf{Дееспособность} --- способность своими осознанными действиями осуществлятьт юридические права и обязанности. \par
        \textbf{Действия} --- определённые законом любые юридические факты, которые являются результатом волевого поведения человека.  \par
        \textbf{Демократия} --- политический режим, дающий гражданам право участвовать в выборе политических решений и выбирать своих представителей в органы власти. \par
        \textbf{Доверенность} --- письменное полномочие, выдаваемое одним лицом(доверителем) другому лицу(доверенному) для представительства перед третьими лицами. \par
        \textbf{Закон} --- нормативный акт, принятый высшим законодательным органом государственной власти либо прямым волеизъявлением населения (путём референдума) и регулирующий наиболее важные общественные отношения. \par
        \textbf{Законодательство} --- весь комплекс(совокупность) нормативных актов, действующих в стране; представляет собой единую систему, части которой(законы, подзаконные акты, отрасли и институты права) взаимодействуют на основе соподчинения(иерархии). \par
        \textbf{Институт права} --- подразделение внути отрасли права --- обособленная группа норм праваЮ регулирующих комплекс взаимосвязанных между собой однородных отношений. \par
        \textbf{Конституция} ---  основной закон государства, нормативный акт, обладающий высшей юридической силоей, определяющий основы государственного строя, организацию государственной власти, отношения государства и гражданина.\par
        \textbf{Кража} --- тайное хищение чужого имущества. \par
        \textbf{Необходимая оборона} --- правомерная защита от общественно опасного посягательства путём причинения вреда нападавшему. Условиями необходимой обороны являются: общественная опасность посягательства, его реальность, причинение вреда только нападающему, соразмерность защиты нападению. \par
        \textbf{Норма права} --- установленное и охраняемое государством общее правило поведения, регулирующее общественные отношения, поведение людей. \par
        \textbf{Нотариат} --- система органов, на которые возложено совершение нотариальных действий:удостоверение сделок, оформление наследственных прав и совершение других действий, юридическое закрепление гражданских прав и предупреждение их возможного нарушения. \par
        \textbf{Основы конституционного строя РФ} --- основные принципы, которые регулируют все стороны жизни общества и государства, определяют правовое положение личности. \par
        \textbf{Отрасль права} --- основное подразделение системы права --- совокупность взаимосвязанных  правовых норм, регулирующих отдельную сферу близких по своему характеру однородных общественных отношений (государственных, трудовых, административных, имущественных, семейных и др.) \par
        \textbf{Подзаконный акт} --- нормативный документ органа государственной власти (указ, распоряжение, поставновление, приказ, инструкция, указание и др.), имеющий более низкую юридическую силу, чем закон: принимается только на основании и во исполнение закона. \par
        \textbf{Политика} --- сфера деятельности, связанная с отношениями между социальными группами, основными проблемами которой является проблема завоевания и использования государственнйо власти. \par
        \textbf{Политическая власть} --- способность и возможность вповодить определённую политику, используя политические партии, организации, государство. \par
        \textbf{Политическая жизнь} --- различные формы взаимодействия участников политики, связанные с борьбой за власть, с выработкой и принятием государственных решений.  \par
        \textbf{Политическая партия} --- организация, объединяющая наиболее активную часть определённых социальных групп или населения в целом, выражающая и защищающая соответствующие социальные интересы. \par
        \textbf{Политический режим} --- совокупность средств и методов осуществления политической власти. \par
        \textbf{Политический экстремизм} --- приверженность некоторых участников политической жизни к крайним взглядам и действиям(насильственным, провокационным и т.п.) в политике. \par
        \textbf{Права ребёнка} --- права человека применительно к детям. \par
        \textbf{Права человека} --- естественная мера свободы и ответственности человека --- нормы, выражающие естественную(прирождённую, неотъемлемую) возможность человека свободно действовать в соответствии со своими интересами, претендовать на достойные условия жизни; объективно необходимы каждому для нормального, полноценного развития личности, участия во всех сферах жизни общества. \par
        \textbf{Право} --- мера свободы, справедливости и ответственности --- вся совокупность установленных государством общеобязательных правил(норм) поведения, исполнение которых обеспечивается силой государственного принуждения. \par
        \textbf{Правовое государство} --- государство, в котором: a) высшей целью является обеспечение прав человека и гражданина; б) государственная власть связана, ограничена правом. \par
        \textbf{Правомерные действия} --- действия, соответствующие требованиям закона. \par
        \textbf{Правонарушение} --- любое деяние (действие или бездействие), нарушающее какие-либо нормы права. \par
        \textbf{Правоотношение} --- социальное отношение, регулируемое нормами права, участники которого имеют юридические права и обязанностиЮ обеспечиваемые силой государства. \par
        \textbf{Правоохранительные органы} --- специальные органы (государственные и негосударственные), создаваемые в целях охраны права, действующие на основании и в соответствии с законом, большинство из них наделены правом применения мер принуждения. \par
        \textbf{Правоспособность} ---  способность иметь права и обязанности.\par
        \textbf{Презумпция невиновности} --- юридический принцип, согласно которому преступный считается невиновным, пока его вина не доказана в установленном законом порядке. \par
\end{document} 
