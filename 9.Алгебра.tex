\documentclass[a4paper, 12pt]{article}
\usepackage[T2A]{fontenc}
\usepackage[utf8]{inputenc}
\usepackage[english, russian]{babel}
\usepackage{indentfirst}
\usepackage{cmap}
\usepackage[lmargin={1cm}, rmargin={1cm}]{geometry}
\usepackage{multirow}
\title{Алгебра.9 класс}
\author{meklomanik(Михаил Колесников)}
\date{Лето 2021}
\begin{document}
    \maketitle{} \clearpage 
    \tableofcontents \clearpage
    \section{Неравенства}
        \subsection{Линейные неравенства с одним неизвестным}
            \subsubsection{Неравенства первой степени с одним неизвестным}
            \subsubsection{Применение графиков к решению неравенств первой степени с одним неизвестным}
            \subsubsection{Линейные неравенства с одним неизвестным}
            \subsubsection{Системы линейных неравенств с одним неизвестным}
            \subsubsection{Неравенства, содержащее неизвестное под знаком модуля}
        \subsection{Неравенства второй степени с одним неизвестным}
            \subsubsection{Понятие неравенства второц степени с одним неизвестным}
            \subsubsection{Неравенства второй степени с положительным дискриминантом}
            \subsubsection{Неравенства второй степени с дискриминантом равным нулю}
            \subsubsection{Неравенства второй степени с отрицательным дискриминантом}
            \subsubsection{Неравенства, сходящиеся к неравенствам второй степени}
        \subsection{Рациональные неравенства}
            \subsubsection{Метод интервалов}
            \subsubsection{Решение рациональных неравенств}
            \subsubsection{Системы рациональных неравенств}
            \subsubsection{Нестрогие неравенства}
            \subsubsection{Замена неизвестного при решении неравенств}
    \section{Степень числа}
        \subsection{Функция y=x^2}
            \subsubsection{Свойства и график функции y=x^n, x>=0}
            \subsubsection{Свойства и график функции y=x^2m, y=x^(2m+1)}
    

\end{document}