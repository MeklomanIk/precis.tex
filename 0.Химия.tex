\documentclass[a4paper, 12pt]{article}
\usepackage[T2A]{fontenc}
\usepackage[utf8]{inputenc}
\usepackage[english, russian]{babel}
\usepackage{indentfirst}
\usepackage{cmap}
\usepackage[lmargin={0.5cm}, rmargin={0.5cm}, bmargin={1cm}, tmargin={1cm}]{geometry}
\usepackage{multirow}
\usepackage{array}
\usepackage{booktabs}
\newcommand{\tab}{~~\llap{\textbullet}~~}
\newcommand{\tbf}{\textbf}
\begin{document}
        \title{Химия.9 класс}
        \author{meklomanik(Михаил Колесников)}
        \date{Лето-Осень 2021}
        \maketitle \clearpage\tableofcontents \clearpage
\section{Общая характеристика химических элементов и химических реакций}    
		
        \begin{table}[!ht]
	    \centering
        \caption{Формы существования химического элемента и их свойства}
        \begin{tabular}{|l|l|c|c|}\hline
            \multicolumn{2}{|l|}{\multirow{2}{*}{\textbf{\begin{tabular}[c]{@{}l@{}}Хим. эл.\end{tabular}}}} & \multicolumn{2}{c|}{\textbf{Изменения свойств}} \\ \cline{3-4} 
            \multicolumn{2}{|l|}{} & \textbf{в главных подгруппах} & \textbf{в периодах} \\ \hline
            \multirow{8}{*}{Атомы} & заряд ядра & $\Uparrow$ & $\Uparrow$ \\ \cline{2-4} 
             & свободные энерг. уровней & $\Uparrow$ & const \\ \cline{2-4} 
             & электроны на внешнем уровне & const & $\Uparrow$ \\ \cline{2-4} 
             & радиус атома & $\Uparrow$ & $\Downarrow$ \\ \cline{2-4} 
             & восстановительные свойства & $\Uparrow$ & $\Downarrow$ \\\cline{2-4} 
             & окислительные свойства & $\Downarrow$ & $\Uparrow$ \\\cline{2-4} 
             & высшая степень окисления & const & $\Uparrow$ \\ \cline{2-4} 
             & низшая степень окисления & const & $\Uparrow$ \\ \hline
            \multirow{2}{*}{\begin{tabular}[c]{@{}l@{}}Простые\\ вещества\end{tabular}} & металические свойства & $\Uparrow$ & $\Downarrow$ \\ \cline{2-4} 
             & неметалические свойства & $\Downarrow$ & $\Uparrow$ \\ \hline
            \begin{tabular}[c]{@{}l@{}}Соединения\\ химических\\ элементов\end{tabular} & \begin{tabular}[c]{@{}l@{}}характер химических свойств\\ высшего оксида и\\ высшего гидроскида\end{tabular} & \begin{tabular}[c]{@{}l@{}}усиление основных\\ свойств и ослабление\\ кислотных свойств\end{tabular} & \begin{tabular}[c]{@{}l@{}}усиление кислотных\\ свойств и ослабление\\ основных свойств\end{tabular} \\ \hline
        \end{tabular}
        \end{table}

        \begin{table}[!ht]
        \centering
                        \caption{Металические и неметаллические свойства атомов в пределах группы и периода}
                        \begin{tabular}{|c|c|c|} \hline
                             \textbf{Свойства}&\textbf{В пределах группы}:& \textbf{В пределах периода}:  \\\hline
                             Мет. свойства & $\uparrow$ & $\downarrow$\\\hline
                             Немет. свойства & $\downarrow$ & $\uparrow$ \\\hline
                             заряды атомных ядер & $\uparrow$ & $\uparrow$ \\\hline
                             число электронов на внешнем уровне & const & $\uparrow$ \\\hline
                             число энергетических уровней & $\uparrow$ & const \\\hline
                             радиус атома & $\uparrow$ & $\downarrow$ \\\hline
                    \end{tabular}
                \end{table}
        
        \begin{table}[!ht]
        \centering
            \caption{Характеристика сложных химических элементов}
            \caption{Характеристика простых химических элементов}
            \begin{tabular}{cccccc|} \\\hline
                \multicolumn{6}{|c|}{\textbf{Кислоты}}\\\hline
                    \multicolumn{2}{| c |}{I - основные} & \multicolumn{2}{| c |}{II - основные} & \multicolumn{2}{| c |}{III - основные} \\\hline
                    \multicolumn{3}{| c |}{Растворимые} & \multicolumn{3}{| c |}{Нерастворимые} \\\hline
                    \multicolumn{3}{| c |}{Кислород-содержащие} & \multicolumn{3}{| c |}{Безкислородные} \\\hline
                    \multicolumn{3}{| c |}{Летучие} & \multicolumn{3}{| c |}{Нелетучие} \\\hline
                    \multicolumn{3}{| c |}{Сильные} & \multicolumn{3}{| c |}{Слабые} \\\hline
                    \multicolumn{3}{| c |}{Стабильные} & \multicolumn{3}{| c |}{Нестабильные} \\\hline
                \multicolumn{6}{| c |}{\textbf{Основания}} \\\hline
                    \multicolumn{3}{| c |}{Растворимые} & \multicolumn{3}{| c |}{Нерастворимые} \\\hline
                    \multicolumn{3}{| c |}{Сильные} & \multicolumn{3}{| c |}{Слабые} \\\hline
                    \multicolumn{3}{| c |}{I - кислотные} & \multicolumn{3}{| c |}{II - кислотные} \\\hline
                \multicolumn{6}{| c |}{\textbf{Соли}} \\\hline
                    \multicolumn{2}{| c |}{Кислые} & \multicolumn{2}{| c |}{Средние} & \multicolumn{2}{| c |}{Основные} \\\hline
                    \multicolumn{3}{| c |}{Растворимые} & \multicolumn{3}{| c |}{Нерастворимые} \\\hline
                \multicolumn{6}{| c |}{\textbf{Оксиды}} \\\hline
                    \multicolumn{2}{| c |}{Основные} & \multicolumn{2}{| c |}{Амфотерные} & \multicolumn{2}{| c |}{Кислотные} \\\hline
                    \multicolumn{3}{| c |}{Растворимые} & \multicolumn{3}{| c |}{Нерастворимые} \\\hline
            \end{tabular}
            \begin{tabular}{|l|}\hline
                \begin{tabular}[c]{@{}l@{}}\textbf{Металлы} \& \textbf{Неметаллы} \\\end{tabular} \\\hline
                \begin{tabular}[c]{@{}l@{}}\tab порядковый номер \\\tab номер группы, вид группы \\\tab номер периода, вид периода\end{tabular}\\\hline
                \begin{tabular}[l]{@{}l@{}}\tab вид простого химического элемента\\\end{tabular}\\\hline
                \begin{tabular}[l]{@{}l@{}}\tab соседи по группе \\\tab развитие мет./немет. свойств\\\end{tabular}\\\hline
                \begin{tabular}[l]{@{}l@{}}\tab соседи по периоду \\\end{tabular}\\\hline
                \begin{tabular}[l]{@{}l@{}}\tab развитие мет./немет. свойств \\\end{tabular}\\\hline
                \begin{tabular}[l]{@{}l@{}}\tab оксид хим.эл., его тип\\\end{tabular}\\\hline
                 \begin{tabular}[l]{@{}l@{}}\tab гидроксид хим.эл., его тип\\\end{tabular}\\\hline
                  \begin{tabular}[l]{@{}l@{}}*\tab высшее летучее соединение с водородом\\\end{tabular}\\\hline
            \end{tabular}
        \end{table}
        \begin{table}[!ht]
        	\centering
            \caption{Признаки веществ}
            \begin{tabular}{| c | c | c |} \hline
                \textbf{Элемент} &  \textbf{Катализатор} & \textbf{Признак} \\\hline
                $Aq^{+}$ & $Cl^{-}$ & белый осадок \\\hline
                $Cu^{2+}$ & $OH^{-}$ & голубой осадок \\\hline
                $Cu^{2+}$ & $S^{2-}$ & чёрный осадок \\\hline
                $Fe^{2+}$ & $OH^{-}$ & зеленоватый, буреющий осадок \\\hline
                $Fe^{3+}$ & $OH^{-}$ & бурый осадок \\\hline
                $Zn^{2+}$ & $OH^{-}$ & белый осадок, OH-растворим. \\\hline
                $Al^{3+}$ & $OH^{-}$ & белый гелеобразный осадок, OH-растворим \\\hline
                $NH_{4}^{+}$ & $OH^{-}$ & запах аммиака \\\hline
                $Ba^{2+}$ & $SO_{4}^{2-}$ & белый осадок \\\hline
                $Ba^{2+}$ & $\Delta$ & жёлто-зелёное пламя \\\hline
                $Ca^{2+}$ & $SO_{4}^{2-}$ & белый осадок \\\hline
                $Ca^{2+}$ & $\Delta$ &кирпично-красное пламя \\\hline
                $Na^{+}$ & $\Delta$ & жёлтое пламя \\\hline
                $K^{+}$ & $\Delta$ & фиолетовое пламя \\\hline
                $Cl^{-}$ & $Ag^{+}$ & белый осадок \\\hline
                $Br^(-)$ & $Ag^{+}$ & желтоватый осадок\\\hline
                $I^{-}$ & $Ag^{+}$ & жёлтый осадок\\\hline
                $SO_{3}^{2-}$ & $H^{+}$ & $SO_{2} \uparrow$ \\\hline
                $CO_{3}^{2-}$ & $H^{+}$ & $CO_{2} \uparrow$ \\\hline
                $NO_{3}$ & $H_{2}SO_{4} + Cu$ & бурый газ\\\hline
                $SO_{4}^{2-}$ & $Ba^{2+}$ & белый осадок\\\hline
                $PO_{4}^{3-}$ & $Ag^{+}$ & жёлтый осадок\\\hline
            \end{tabular} 
        \end{table}
	    \begin{table}[!ht]
	    \centering
	    \caption{Генетические ряды}
	    \begin{tabular}{|llll|}
	    \hline
	    \multicolumn{4}{|c|}{\textbf{Металл}} \\ \hline
	    \begin{tabular}[c]{@{}l@{}}(простое\\ вещество)\end{tabular} & $\rightarrow$ основный оксид & $\rightarrow$ основание & $\rightarrow$ соль \\ \hline
	    \multicolumn{4}{|c|}{\textbf{Неметалл}} \\ \hline
	    \begin{tabular}[c]{@{}l@{}}(простое\\ вещество)\end{tabular} & $\rightarrow$ кислотный оксид & $\rightarrow$кислота & $\rightarrow$ соль \\ \hline
	    \end{tabular}
	    \end{table}

	    \begin{table}[!ht]\centering
	    \caption{Оксиды и гидроксиды амфотерных веществ на примере Cr}
	    \begin{tabular}{|ccc|}\hline
	    \multicolumn{3}{|c|}{$Cr$} \\ \hline
	    \multicolumn{1}{|c|}{\begin{tabular}[c]{@{}c@{}}$Cr^{+2}O$\\ ---  основный оксид хрома(II)\end{tabular}} & \multicolumn{1}{c|}{\begin{tabular}[c]{@{}c@{}}$Cr_{2}^{+3}O_{3}$\\---  амфотерный оксид хрома(III)\end{tabular}} & \begin{tabular}[c]{@{}c@{}}$Cr^{+6}O_{3}$\\ ---  кислотный оксид хрома(IV)\end{tabular} \\ \hline
	    \multicolumn{1}{|c|}{\begin{tabular}[c]{@{}c@{}}$Cr(OH)_{2}$\\ ---  основание\end{tabular}} & \multicolumn{1}{c|}{\begin{tabular}[c]{@{}c@{}}$Cr(OH)_{3}$ или $HCrO_{2}$\\ --- амфотерный гидроксид\end{tabular}} & \begin{tabular}[c]{@{}c@{}}$H_{2}Cr0_{4}$ или $H_{2}Cr_{2}O_{7}$\\ --- 	 кислоты\end{tabular} \\ \hline
	    \end{tabular}
	    \end{table}

	    

	    
	    
\section{Металлы}
    	\subsection{Металлические века}
    	\subsection{Положение металлов в П.С..Строение их атомов}
    		Разделение химических элементов на металлы и неметаллы условно. 
    		Металлы как вещества могут быть только восстановителями.
    		Исключая амфотерные вещества (условную границу между металлами и неметаллами B --- Si --- As --- Te --- At):
    		Металлы I-ой группы --- щелочные металлы.
    		Металлы II-ой группы --- щелочноземельные металлы.
    	\subsection{Физические свойства металлов}
    		Метталическая связь обуславливает все физические свойства металлов:
    		
    		\tab металический блеск
    		
    		\tab твёрдость
    		
    		\tab плотность
    		
    		\tab t плавления
    	    
\section{Определения}
        Химия --- наука о веществах, их свойствах и превращениях.
        
        Свойства веществ  --- признаки отличия веществ.

		Химический элемент --- совокупность атомов с одинаковым зарядом ядер.

		Переходные элементы, переходные металлы --- элементы побочной полгруппы П.с. образующие амфотерные оксиды и гидроксиды.
        
        Металлы --- химические элементы, атомы которых стремятся отдать электроны с внешнего электронного уровня.
        
        Неметаллы --- химические элементы, атомы которых стремятся принять электроны с внешнего электронного уровня.
        
        Оксиды --- сложные вещества, состоящие из химических элементов один из которых О$^{-2}$.
        
        Несолеобразующие оксиды -- оксиды, которые образуют кислоты и щёлочи, не образуют соли.
        
        Основные оскиды --- оксиды, которые соответствуют (образуют) основания.
        
        Кислотные оксиды --- оксиды, которые соответствуют кислотам.

       	Амфотерные оксиды --- оксиды, которые соответствуют кислотам и основаниям.
        
        Аллотропия --- способность атомов образовывать несколько простых веществ.
        
        Химическое уравнение --- условная запись химической реакции с помощью химических формул и математических знаков.

        Гомогенный --- одного типа, вида, рода.

        Гетерогенный -- разного типа, вида, рода.

        Катализаторы --- не участвующие в реакции вещества,но ускоряющие её или изменяющие пути её течения.

        Ферменты --- биологические катализаторы белковой природы.

        Катализ --- процесс изменения скорости химической реакции, или пути её течения.
        
        Качественные реакции --- р-ии с определением вещества.
        
        Горения реакции --- р-ии, протекающие с выделением телпа и света.
      
        Экзотермические реакции --- р-ии с выделением тепла.
        
        Эндотермические реакции --- р-ии с поглощением тепла.
        
        Разложения реакции --- р-ии со сложным веществом разлагающимся на несколько простых веществ.
        
        Соединения реакции --- с двумя сложными веществами образующими два новых сложных вещества.
        
        Замещения реакции --- реакции с замещением атомов простого на атомы сложного вещества.
        
        Обмена реакции --- реакции с взаимо-замещением атомов двух сложных веществ.
        
        Ионная связь --- связь между ионами.
        
        Атомная связь --- связь в результате образования электронной пары.
        
        Метталическая связь --- связь между атом-ионами в металлах и сплавах за счёт обобществлённых электронов.
        
        Окисление --- процесс отдачи электронов.
        
        Восстановление --- процесс принятия электронов.
        
        Окислитель --- частица, принимающая электроны.
        
        Восстановитель --- частица, отдающая электроны.
        
        Пирометалургия --- восстановление металлов с помощью реакций, возникающих при высоких температурах.
        
        Молярная масса --- отношение массы вещества к количеству.

        Концентрация --- отношение количества вещества к занимаемому объёму.
        
        Насыщенный раствор --- вещество больше не растворяется.
        
        Ненасыщенный раствор --- вещество растворилось, остался раствор
        
        Пересыщенный раствор --- вещество растворилось, осталось вещество.
        
        Электролиты --- проводящие электрический ток вещества.
        
        Неэлектролиты --- не проводящие электрический ток вещества.
        
        Электролитическая диссоциация (Э.Д.) --- процесс распада электролита на ионы.
        
        Степень диссоциация --- отношение количества электролита распавшегося на ионы, к общему количеству.
        	    
      	Скорость химической реакции --- изменение концентрации реагирующих веществ в единицу времени: $V_{p}=C_{1}-C_{2}/t$ .

\section{Законы}
		
        \tbf{Периодический закон} --- свойства химических элементов и образованных ими веществ находятся  в периодической зависимости от зарядов их ядер.

        
\end{document}