\documentclass[a4paper, 12pt]{article}
\usepackage[T2A]{fontenc}
\usepackage[utf8]{inputenc}
\usepackage[english, russian]{babel}
\usepackage{indentfirst}
\usepackage{cmap}
\usepackage[lmargin={1cm}, rmargin={1cm}]{geometry}
%\usepackage{pgfpages}
\title{Обществознание.8 класс}
\author{meklomanik(Михаил Колесников)}
\date{Лето 2021}
\begin{document}
    \maketitle{}
    
\clearpage
\tableofcontents
\section{Экономика}
\subsection{Экономика и её роль в жизни общества}
Потребности --- нужда в чём либо необходимом для поддержания жизнедеятельности и развития личности, группы людей и общества в целом.
Цель удовлетворения потребностей - создание условий для жизни и деятельности людей.
Ограниченность ресурсов - недостаточность имеющихся в распоряжении людей ресурсов для производства благ, способных удовлетворить возрастающие потребности человека и общества.
Благо - средство удовлетворения потребностей.
Жизненные блага:
\begin{itemize}
    \item Свободные --- доступны для всех нуждающихся в них.
    \item Экономические --- ограничены.
\end{itemize}
Производство экономических благ --- основа жизни человеческого общества.
Альтернативная стоиость - польза или выгода, которую мы могли бы получить от самого лучшего из невыбранных вариантов.
\subsection{Рыночная экономика}
 Характерные черты: 
 \begin{itemize}
     \item свободный обмен между продавцами и покупателями
     \item право частной собственности на экономические ресурсы
     \item материальная ответственность участников
 \end{itemize}
 Условия для функционирования:
 \begin{itemize}
     \item свобождное ценообразование
     \item свобода предпринимательской деятельности
 \end{itemize}
 Конкуренция --- соперничество, борьба за достижение лучших результатов.
 Конкуренция --- мотор рыночной экономики
 Способы экономической конкуренции:
 \begin{itemize}
     \item снижение цены
     \item улучшение качества товаров
     \item реклама 
 \end{itemize}
Пиар --- реклама товара, без товара.
Невозможность влияния на уровень цены:
\begin{itemize}
    \item попытки повысить цену
    \item искусственное снижение цены
\end{itemize}
Рыночный механизм:
\begin{itemize}
    \item спрос
    \item предложение 
    \item цена
\end{itemize}
\end{document}